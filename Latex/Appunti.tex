\documentclass[12pt, a4paper]{report}
\usepackage[utf8]{inputenc}
\newcommand\preamble{
    \usepackage[italian]{babel}
    \usepackage{geometry}
    \usepackage{amsmath}
    \usepackage{amssymb}
    \usepackage{graphicx}
    \usepackage[dvipsnames]{xcolor}
    \usepackage{ulem}
    \usepackage{listings}
    \usepackage{tikz}
    \usetikzlibrary{shapes.geometric, arrows}
    \usepackage{array}
    \usepackage{varwidth}
    
    \let\olditemize\itemize
    \renewcommand\itemize{\olditemize\setlength\itemsep{0em}}

    \tikzstyle{process} = [rectangle, minimum width=3cm, minimum height=1cm, text centered, text width=3cm, draw=black, fill=orange!30]
    \tikzstyle{decision} = [diamond, minimum width=3cm, minimum height=1cm, text centered, text width=2cm, draw=black, fill=green!30]
    \tikzstyle{arrow} = [thick,->,>=stealth]
    \geometry{left=1cm, right=1cm, top=2cm, bottom=2cm}
}
\newcommand{\customfbox}[1]{
    \begin{center}
        \fbox{\parbox{\dimexpr\linewidth-2\fboxsep-2\fboxrule\relax}{\centering #1}}
    \end{center}
}

% Definizione delle variabili
\newcommand{\imagePath}{Immagini/logoUni.png}

% Definizione del comando per la pagina di titolo con argomenti
\newcommand{\customTitlePage}[2]{
    \newcommand{\courseTitle}{#1}
    \newcommand{\academicYear}{#2}
    
    \begin{titlepage}
        \centering
        \includegraphics[width=0.5\textwidth]{\imagePath}\par\vspace{1cm}
        {\scshape\LARGE Università degli Studi di Genova \par}
        \vspace{1.5cm}
        {\huge\bfseries \courseTitle \par}
        \vspace{2cm}
        {\Large\itshape Lorenzo Vaccarecci \par}
        \vfill
        \academicYear
    \end{titlepage}
}


\preamble

\begin{document}
\customTitlePage{Fondamenti di Ingegneria del Software}{2024-2025}
\newpage
\tableofcontents
\chapter{Modelli di processo di sviluppo software}
\section{Introduzione}
\customfbox{\textbf{Processo}: insieme strutturato e organizzato di attività che si svolgono per ottenere un risultato.}
Perchè modellare il processo? Per dare ordine, controllo e ripetibilità con l'intenzione di migliorare la produttività e la qualità del prodotto.
\subsection{Processo prescrittivo e adattivo}
\begin{itemize}
    \item \textbf{Processo prescrittivo}: un processo che segue un modello predefinito e rigido, con passaggi specifici e ben definiti.
    \item \textbf{Processo adattivo}: un processo che permette modifiche e adattamenti durante il suo svolgimento.
\end{itemize}
Perchè studiare  i modelli di processo? Perchè uno dei compiti dei manager aziendali è quello di decidere il modello di processo da adottare considerando la tipologia del software da progettare e il personale disponibile.
\section{Modelli di processo}
\section{Code and Fix}
\begin{minipage}{0.4\textwidth}
\begin{itemize}
    \item Si arriva al codice finale "per tentativi"
    \item Non adatto per progetti grandi con tanti sviluppatori
    \item \textcolor{red}{Non è un modello di processo vero e proprio}
\end{itemize}
\end{minipage}
\hspace{0.05\textwidth}
\begin{minipage}{0.5\textwidth}
    \begin{tikzpicture}[node distance=2cm]
        \node (start) [process] {Capire i requisiti};
        \node (process1) [process, below of = start] {Codifica da zero, esecuzione e test};
        \node (dec1) [decision, below of = process1, yshift = -1cm] {Passano i test?};
        \node (process2a) [process, below of = dec1, yshift = -1cm] {Aggiustamento del codice, ri-esecuzione e test};
        \node (stop) [process, left of = dec1, xshift = -2cm] {Consegna del codice};
        \draw [arrow] (start) -- (process1);
        \draw [arrow] (process1) -- (dec1);
        \draw [arrow] (dec1) -- node[anchor=east] {no} (process2a);
        \draw [arrow] (dec1) -- node[anchor=south] {si} (stop);
        \draw [arrow] (process2a.east) -| ++(1,0) |- ([yshift=-0.3cm]process1.south); 
    \end{tikzpicture}
\end{minipage}
\section{Modello a cascata}
\begin{minipage}{0.5\textwidth}
\begin{itemize}
    \item \textcolor{red}{Storicamente il primo modello del processo di sviluppo software}
    \item Ogni fase produce un prodotto che è l'input della fase successiva
    \item Con il modello waterfall abbiamo il passaggio dalla dimensione artigianale alla produzione industriale del software
    \item Molto rigido: non si può tornare indietro
\end{itemize}
\end{minipage}
\hspace{0.05\textwidth}
\begin{minipage}{0.4\textwidth}
    \includegraphics[width=\linewidth]{Immagini/waterfall.png}
\end{minipage}
\begin{center}

\begin{tabular}{|p{0.45\textwidth}|p{0.45\textwidth}|}
    \hline
    \textbf{\textcolor{Green}{Vantaggi}} & \textbf{\textcolor{red}{Svantaggi}} \\
    \hline
    Enfasi su aspetti come l'analisi dei requisiti e il progetto di sistema trascurati nell'approccio code \& fix & Lineare, rigido, monolitico: no feedback tra fasi, no parallelismo, \textcolor{red}{unica data di consegna} \\
    \hline
    Pospone l'implementazione dopo avere capito i bisogni del cliente & La consegna avviene dopo anni, intanto i requisiti cambiano o si chiariscono: così viene consegnato software obsoleto \\
    \hline
    Introduce disciplina e pianificazione & Viene prodotta troppa documentazione poco chiara: l'utente spesso non conosce tutti i requisiti all'inizio dello sviluppo \\
    \hline
    E' applicabile se i requisiti sono chiari e stabili & Alcuni difetti superati da modello waterfall con feedback e iterazioni \\
    \hline
\end{tabular}
\end{center}
\subsection{Studio di fattibilità}
\begin{itemize}
    \item Fase che precede lo sviluppo vero e proprio
    \item Viene analizzata la fattiblità e convenienza del progetto
    \item Stima dei costi
    \item Si valuta il Return Of Investment (ROI)
\end{itemize}
\subsection{Varianti del modello a cascata}
\begin{itemize}
    \item Cascata con prototipazione: prima di iniziare lo sviluppo si costruisce un prototipo "usa e getta" con il solo scopo di fornire agli utenti una base concreta per meglio definire i requisiti.
    \item Cascata con feedback e iterazioni: posso tornare a una fase precedente. \begin{center}
        \includegraphics[width=0.8\linewidth]{Immagini/waterfallfeedback.png}
    \end{center}
    \item V-Model: \begin{itemize}
        \item Enfasi sulle fasi di testing
        \item Evidenzia come le attività di testing (parte destra della V) sono collegate a quelle di analisi e progettazione (parte sinistra della V)
        \item Ogni controllo fatto a destra che non dia buon esito porta a un rifacimento/modifica di quanto fatto a sinistra
        \item \textbf{Parallelismo}: creazione dei test e una volta che ho il codice li eseguo
        \item \textcolor{red}{Problemi (anche per Waterfall)}: \begin{itemize}
            \item Versione funzionante solo alla fine!
            \item Errore in fase iniziale può avere conseguenze disastrose
        \end{itemize}
    \end{itemize}
    \begin{center}
        \includegraphics[width=0.5\linewidth]{Immagini/vmodel.png}
    \end{center}
\end{itemize}
\section{Modelli evolutivi}
\customfbox{Idea: sviluppare un implementazione iniziale, esporla agli utenti e raffinarla attraverso successivi rilasci del SW (release)}
Sottocategorie:
\begin{itemize}
    \item Prototyping
    \item Modelli incrementali
    \item Modelli iterativi
\end{itemize}
\subsection{Modelli a Prototyping}
\begin{itemize}
    \item Realizzazione di un prototipo funzionante del sistema, su cui validare i requisiti (o l'architettura)
    \item Il prototipo ha meno funzionalità ed è meno efficiente
\end{itemize}
\begin{center}
    \begin{tabular}{|p{0.45\textwidth}|p{0.45\textwidth}|}
        \hline
        \textbf{\textcolor{Green}{Vantaggi}} & \textbf{\textcolor{red}{Svantaggi}} \\
        \hline
        Permette di raffinare requisiti definiti in termini di obiettivi generali e troppo vaghi & Il prototipo è un meccanismo per identificare i requisiti, spesso da "buttare": problema economico e psicologico, il rischio è di non farlo e così scelte non ideali diventano parte integrante del sistema \\
        \hline
        Rilevazione precoce di errori di interpretazione & \\
        \hline
    \end{tabular}
\end{center}
\subsection{Modelli Iterativi-Incrementali}
\begin{itemize}
    \item Sviluppo di varie release,  di cui solo l'ultima è completa
    \item Dopo la prima release, si procede in parallelo
    \item Le fasi di sviluppo vengono percorse più volte
\end{itemize}
\begin{center}
    \includegraphics[width=0.5\linewidth]{Immagini/iterativoincrementale.png}
\end{center}
\subsubsection{Modelli Incrementali}
\begin{itemize}
    \item Ogni release aggiunge nuove funzionalità
    \item Nella fase di pianificazione si decide il requisito/funzionalità da includere nella release successiva.
    \item Si trattano per prime le funzionalità ad alto rischio
    \item Si cerca di massimizzare il valore per gli utenti
\end{itemize}
\subsubsection{Modelli Iterativi}
\begin{itemize}
    \item Da subito sono presenti tutte (o buona parte) delle funzionalità che sono via via raffinate, migliorate
\end{itemize}
\section{Modello a spirale}
\begin{itemize}
    \item Sistemi di grandi dimensioni
    \item Approccio "evolutivo" con interazioni continue fra cliente e developer
    \item Modello "risk-driver": tutte le scelte sono basate sui risultati dell'analisi dei rischi
    \item 'Meta-modello': dà un'idea generale ma quando si inizia a lavorare bisogna scegliere un modello esistente \begin{itemize}
        \item Requisiti chiari e stabili $\rightarrow$ modello a cascata
        \item Requisiti confusi $\rightarrow$ prototipo
    \end{itemize}
\end{itemize}
\customfbox{\textbf{Rischio}: circostanza potenzialmente avversa in grado di pregiudicare lo sviluppo e la qualità del software}
\textcolor{red}{Ogni scelta/decisione ha un rischio associato}, due caratteristiche importanti nella valutazione di un rischio sono: \begin{itemize}
    \item Gravità delle conseguenze
    \item Probabilità che si verifichi la circostanza
\end{itemize}
\begin{center}
    \includegraphics[width=0.7\linewidth]{Immagini/spirale.png}
\end{center}
\begin{itemize}
    \item \textbf{Planning}: determinazione di obbiettivi, alternative, vincoli
    \item \textbf{Risk Analysis}: analisi delle alternative e identificazione/risoluzione dei rischi
    \item \textbf{Engineering}: sviluppo del prodotto di successivo livello
    \item \textbf{Customer Evaluation}: valutazione dei risultati dell'engineering dal punto di vista del cliente
\end{itemize}
\begin{center}
    \begin{tabular}{|p{0.45\textwidth}|p{0.45\textwidth}|}
        \hline
        \textbf{\textcolor{Green}{Vantaggi}} & \textbf{\textcolor{red}{Svantaggi}} \\
        \hline
        Adatto allo sviluppo di sistemi complessi & Non è un rimedio universale (panacea) \\
        \hline
        Primo approccio che considera il rischio (risk-driver) & Necessita competenze di alto livello per la stima dei rischi \\
        \hline
        & Richiede  un'opportuna personalizzazione ed esperienza di utilizzo \\
        \hline
        & Se un rischio rilevante non viene scoperto o tenuto a bada si inizia da zero \\
        \hline
    \end{tabular}
\end{center}
\section{Unified Process}
\begin{itemize}
    \item Specifico per sistemi ad oggetti, con uso di notazione UML per tutto il processo
    \item Guidato dagli \textbf{Use Case}
    \item Incorpora molte delle idee 'buone' dal modello a spirale
    \item Meta-modello
    \item Supportato da tool(visuali) in ogni fase
    \item Processo prescrittivo per eccellenza
\end{itemize}
\subsection{Le iterazioni}
\begin{itemize}
    \item Possibili diverse iterazioni che terminano con il rilascio del prodotto
    \item Ogni iterazione consiste di quattro fasi (anche ripetute più volte) che terminano con una milestone (= rilascio di artefatti soggetti a controllo)
    \item Ogni fase è costituita da diverse attività: \begin{itemize}
        \item Requisiti (R)
        \item Analisi (A)
        \item Design (D)
        \item Codifica (C)
        \item Testing (T)
    \end{itemize}
\end{itemize}
\subsection{Le fasi}
\begin{itemize}
    \item Inception: studio di fattibilità, requisiti essenziali del sistema, risk analysis
    \item Elaboration: sviluppa la comprensione del dominio e del problema, gli Use Case della release da rilasciare, l'architettura del sistema
    \item Construction: Design (in UML), codifica e testing del Sistema
    \item Transition: Messa in esercizio della release nel suo ambiente (deploy), training e testing da parte di utenti fidati
\end{itemize}
\section{Sviluppo basato sui componenti}
Modello che va nella direzione del \textbf{riutilizzo del software}
\begin{center}
    \begin{tabular}{|p{0.45\textwidth}|p{0.45\textwidth}|}
        \hline
        \textbf{\textcolor{Green}{Vantaggi}} & \textbf{\textcolor{red}{Svantaggi}} \\
        \hline
        Riduce la quantità di software da scrivere & Sono necessari dei compromessi: requisiti iniziali potrebbero differire da quelli che si possono soddisfare con le componenti disponibili \\
        \hline
        Riduce i costi totali di sviluppo e i rischi & Integrazione non sempre facile \\
        \hline
        Consegne più veloci & Spesso i componenti usati sono fatti evolvere dalla ditta produttrice senza controllo di chi li usa \\
        \hline
    \end{tabular}
\end{center}
\section{Metodi Plan-Driven e Agili}
\begin{center}
    \begin{tabular}{|p{0.45\textwidth}|p{0.45\textwidth}|}
        \hline
        \textbf{Plan-Driven} & \textbf{Agile} \\
        \hline
        Seguono un approccio classico dell'ingegneria dei sistemi fondato su processi ben definiti e ocn passi standard & Rispondere ai cambiamenti dei requisiti in modo veloce \\
        \hline
        & Filosofia del programmare come "arte" piuttosto che processo industriale \\
        \hline
        & Cosa più importante soddisfare il cliente e non seguire un piano (contratto)\\
        \hline
    \end{tabular}
\end{center}
\begin{figure}[h]
    \centering
    \includegraphics[width=0.5\linewidth]{Immagini/agilemanifesto.png}
    \caption{The Agile Manifesto}
\end{figure}
\subsection{Come scegliere?}
\textbf{Metodi plan-driven}: \begin{itemize}
    \item Sistemi grandi e comploessi, safety-critical o con forti richieste di affidabilità
    \item Requisiti stabili e ambiente predicibile
\end{itemize}
\textbf{Metodi agili}: \begin{itemize}
    \item Sistemi e team piccoli, clienti e utenti disponibili, ambiente e requisiti volatili
    \item Team con molta esperienza
    \item Tempi di consegna rapidi
\end{itemize}
\newpage
\section{DevOps}
Metodo di sviluppo evolutivo
\begin{figure}[h]
    \centering
    \includegraphics[width=0.8\linewidth]{Immagini/devops.png}
    \caption{DevOps}
\end{figure}
\subsection{Continuous Integration}
La Continuos Integration (CI), o Integrazione Continua, è una pratica di sviluppo software in cui i programmatori integrano frequentemente il proprio lavoro (codice) nel repository condiviso del progetto, in genere diverse volte al giorno.
\chapter{Ingegneria dei requisiti}
\section{Introduzione}
Descrivere 'qualcosa' che il sistema dovrà fare (una funzionalità) o un vincolo a cui deve sottostare
\begin{itemize}
    \item \textbf{Diversi livelli di astrazione}:\begin{itemize}
        \item Descrizione astratta ed imprecisa del sistema
        \item Descrizione dettagliata e matematica dello stesso
    \end{itemize}
\end{itemize}
\customfbox{Che cosa il sistema farà e non come!}
E' importante definire i requisiti in modo da evitare difetti in fasi avanzate del progetto, infatti i difetti dovrebbero essere scoperti il più presto possibile, ovvero a livello dei requisiti.
\section{Classificazione dei requisiti}
\begin{itemize}
    \item \textbf{Requisiti utente}: descrizione in linguaggio naturale delle funzionalità che il sistema dovrà fornire e dei vincoli operativi (\underline{sono scritti per (e con) il cliente})
    \item \textbf{Requisiti di sistema}: descrive in modo dettagliato le funzionalità che il sistema dovrà fornire (\underline{sono scritti per gli sviluppatori})
    \item \textbf{Requisiti funzionali}: descrivono ciò che il sistema dovrà fare, non come ma cosa
    \item \textbf{Requisiti non-funzionali}: definiscono vincoli sul sistema e sullo sviluppo del sistema, in generale riguardano la scelta di linguaggi, piattaaforme, strumenti, tecniche d'implementazione, ma anche: prestazioni, questioni etiche, ...
\end{itemize}
Un requisito etico può essere ad esempio che nella realizzazione dell'applicazione verranno utilizzato solo strumenti e servizi 'non proprietari' (es. no Microsoft)
\subsection{Esempio: Bancomat}
In \textcolor{red}{rosso} i requisiti funzionali, in \textcolor{blue}{blu} i requisiti non funzionali
\begin{itemize}
    \item \textcolor{red}{Il sistema deve mettere a disposizione le funzioni di prelievo, saldo e estratto conto}
    \item \textcolor{blue}{Il sistema deve essere disponibile a persone portatori di Handicap, deve garantire un tempo di risposta inferiore al minuto, e deve essere sviluppato su architettura X86 con sistema operativo compatibile con quello della Banca}
    \item \textcolor{red}{Le operazioni di prelievo devono richiedere autenticazione tramite un codice segreto memorizzato sulla carta}
    \item \textcolor{blue}{Il sistema deve essere facilmente espandibile, e adattabile alle future esigenze bancare}
\end{itemize}
\section{Requirements Engineering}
E' il termine usato per descrivere le attività necessarie per raccogliere, documentare e tenere aggiornato l'insieme dei requisiti di un sistema software.
\subsection{Scopo}
Lo scopo primario del RE è la produzione di un documento (il requirement document) che definisca le funzionalità e i servizi offerti dal sistema da realizzare (anche tenerlo aggiornato)
\subsection{Processo iterativo}
\begin{center}
    \includegraphics[width=0.6\textwidth]{Immagini/reiterativo.png}
\end{center}
\begin{itemize}
    \item \textbf{Elicitation}: \begin{itemize}
        \item Ottenere, estrarre, ricavare, tirar fuori i requisiti dal cliente e da altri partecipanti
        \item Il primo passo è identificare gli stakeholders\footnote{Stakeholder: persona veramente interessata allo sviluppo del progetto}
        \item Intervise, osservazioni sul luogo di lavoro, questionari, analisi dei prodotti dei competitors, workshop (brainstorming)
        \item Studio/analisi di leggi e regolamenti, help-desk reports, 'change requests' di prodotti analoghi, 'lessons learned' in progetti simili, ...
    \end{itemize}
    \item \textbf{Analisi dei requisiti}: \begin{itemize}
        \item I bisogni (user needs) degli stakeholders raccolti durante la fase di elicitation sono analizzati e raffinati
        \item Si cerca di capire se i requisiti sono corretti
        \item Si cercano di identificare i "missing requirements"
        \item Si identificano requisiti poco chiari
        \item Si risolvono i requisiti "contradditori o in conflitto"
        \item Viene stabilità la priorità (prioritizzazione): \begin{itemize}
            \item Per sapere cosa "tagliare" se non tutti potranno essere realizzati
            \item Scala numerica
            \item Scala MoSCoW: \begin{itemize}
                \item \textbf{Must have}: requisiti obbligatori
                \item \textbf{Should have}: requisiti importanti ma non indispensabili
                \item \textbf{Could have}: requisiti desiderabili ma non necessari
            \end{itemize} 
        \end{itemize}
    \end{itemize}
    \item \textbf{Definizione e specifica}: \begin{itemize}
        \item Definizione dei requisiti utente: costituisce un contratto fra le parti
        \item Specifica dei requisiti di sistema: costituisce "starting point" per la fase di design
    \end{itemize}
    \item \textbf{Validazione}: \begin{itemize}
        \item Esame della definizione/specifica dei requisiti per valutarne la qualità
        \item Di solito la convalida o validazione si effettua mediante 'formal peer reviewes'
        \item Scrivere dei casi di test a partire dai requisiti
        \item Sviluppare un prototipo
    \end{itemize}
    \item \textbf{Requirements Management}: \begin{itemize}
        \item Approvazione di alcune richieste di cambio dei requisiti
        \item Negoziazione con il cliente
        \item Impact analysis per i cambi richiesti
        \item Tenere allineati i requisiti e il codice (e casi di test)
        \item Tracciare il progresso di un progetto
    \end{itemize}
\end{itemize}
\section{Proprietà dei requisiti}
\begin{itemize}
    \item \textbf{Validità-correttezza}
    \item \textbf{Consistenza}: non ci sono requisiti contradditori
    \item \textbf{Completezza}: tutti gli aspetti che il cliente vuole sono coperti nei requisiti (in teoria)
    \item \textbf{Realismo}: non si chiede l'impossibile
    \item \textbf{Inequivocabilità (Unambiguos)}: ogni requisito dovrebbe avere solo un interpretazione
    \item \textbf{Verificabilità}: i requisiti vanno espressi in modo che siano testabili
    \item \textbf{Tracciabilità}: \begin{itemize}
        \item Ogni funzionalità implementata nel sistema deve poter essere fatta risalire a dei requisiti in modo semplice
        \item Ogni requisito nella requirement specification deve corrispondere ad uno nella requirement definition
    \end{itemize}
\end{itemize}
\section{Template/Schema dei requisiti}
Conviene attenersi a questo Schema
\begin{center}
    \large
    \texttt{<id>} il \texttt{<sistema>} deve \texttt{<funzione>}
\end{center}
Es. \texttt{R1.}Il sistema deve gestire tutti i regitratori di cassa del negozio (non più di 20)
\section{Analista software}
L'analista software o di sistema è la persona che:
\begin{itemize}
    \item si occupa dell'elicitazione dei requisiti
    \item analizza i requisiti
    \item scrive il documento dei requisiti (definizione e/o specifica)
    \item Comunica/spiega i requisiti a sviluppatori e altri stakeholder
\end{itemize}
Alcune competenze che un analista dovrebbe avere:
\begin{itemize}
    \item Arte della negoziazione
    \item Stabilire una strategia (problem solving)
    \item Giusta capacità di imporsi
    \item Ascoltare attentantemente
    \item Dono della sintesi
    \item Padronanza del linguaggio naturale
    \item Buona conoscenza del dominio (ad esempio in ambito medico o automobilistico)
\end{itemize}
\subsection{Consigli per un'intervista}
\begin{enumerate}
    \item Fare molte domande
    \item Ascoltare bene
    \item Mettere in discussione i quantificatori universali: 'tutto, ogni, sempre, ...'
    \item Annotare tutte le risposte
\end{enumerate}
\subsection{Importanza della comunicazione}
\begin{itemize}
    \item Elicitation = Attività molto delicata perchè mette in comunicazione due o più persone di realtà anche molto diverse
    \item Frequenti incomprensioni, che si ripercuotono sulla qualità dei requisiti
\end{itemize}
Occore fare molta attenzione a:
\begin{itemize}
    \item Diversità di significato che si attribuisce ai termini $\rightarrow$ possibile soluzione definizione del glossario: \begin{itemize}
        \item Per la spiegazione dei termini tecnici
        \item Per ridurre l'ambiguità dei termini usati
        \item Per "espandere" gli acronimi
    \end{itemize}
    \item Assunzioni nascoste (Hidden assumptions)
    \item Verbosità (= sovrabbondanza di parole)
    \item Mancanza di chiarezza/precisione
\end{itemize}
\section{Consigli finali}
\begin{itemize}
    \item Riuso di (parte di) requisiti
    \item Utilizzo di un glossario comune tra clienti, utenti e analisti
    \item Utilizzo di un 'buon' template/form
    \item Utilizzo di un software per la geitone/raccolta e analisi dei requisiti
\end{itemize}
\end{document}